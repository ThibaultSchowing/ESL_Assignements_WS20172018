\documentclass[11pt,a4paper,twoside,openright]{report}

% Packages 

\usepackage[english]{babel}
\usepackage[utf8]{inputenc}

\usepackage{amsmath}
\usepackage{amsfonts}
\usepackage{amssymb}
\usepackage{makeidx}
\usepackage{graphicx}
\usepackage{afterpage}
\usepackage{cite}
\usepackage{longtable}

\usepackage[section]{placeins}
\usepackage{float}
\usepackage{listings}
\usepackage{color}

\usepackage{booktabs} % To thicken table lines
\usepackage{pgfplotstable}
\usepackage[final]{pdfpages}


\usepackage[hidelinks]{hyperref}
%\usepackage{minitoc}

\usepackage{cleveref}

\usepackage{geometry} 



\lstset{frame=tb,
	language=R,
	keywordstyle=\color{blue},
	alsoletter={.}
}

\usepackage{enumitem}
\renewcommand\descriptionlabel[1]{\textbf{#1 :}}

\usepackage{subfig}
\usepackage{graphicx}

\usepackage{array}

% Evite les gros début de chapitre inutiles. 
\usepackage{titlesec}

% niveaux de table des matières
\setcounter{secnumdepth}{3}
\setcounter{tocdepth}{3}


\titleformat{\chapter}
{\normalfont\LARGE\bfseries}{\thechapter}{1em}{}
\titlespacing*{\chapter}{0pt}{3.5ex plus 1ex minus .2ex}{2.3ex plus .2ex}

% Modification de commandes 

\newcommand\blankpage{%
	\null
	\thispagestyle{empty}%
	\addtocounter{page}{-1}%
	\newpage}

%\graphicspath{{img}}


\author{Thibault Schowing and Sarah Mcleod}


\usepackage{fancyhdr,graphicx,lastpage}% http://ctan.org/pkg/{fancyhdr,graphicx,lastpage}
\fancypagestyle{plain}{
	\fancyhf{}% Clear header/footer
	\fancyhead[R]{Thibault Schowing and Sarah Mcleod}% Right header
	\fancyhead[L]{Elements of Stastical Learning WS2017/2018}% Left footer
	\fancyfoot[R]{\thepage}% Right footer
}





\begin{document}	
	% Page de titre
	\pagestyle{plain}% Set page style to plain.

	\newgeometry{hmarginratio=1:1}
	
	\pagenumbering{gobble}
	
	\begin{titlepage}
		\centering
		
		%\vspace{0.5cm}
		\small{Saarland University  \par}
		\huge{The Elements of Stastical Learning\par}
		\vspace{1cm}
		
		
		
		\vspace{1cm}
		\Large{Assignement 1\par}
		\vspace{1.5cm}
		\small{Due Date: 08.11.2017  \par}
		\vspace{1cm}
		
		
		
		\vspace{2cm}
		\small\textit{Thibault \textsc{Schowing}}\par
		\small\textit{Sarah \textsc{Mcleod}}\par
		
		\small{\today\par}
		
		\vfill
		
		
	\end{titlepage}
	% Geometry centrée pour page de titre: off
	\restoregeometry 
	%\afterpage{\blankpage}
	
	
	
	\section*{Problem 1}
	
	Statistical learning is the process of learning from data. The majority of statistical learning can be divided into two categories: \textbf{Supervised Learning} and \textbf{Unsupervised learning}. In Supervised Learning, you have \textbf{inputs} as well as the corresponding \textbf{outcomes} that serve to direct the learning process. The goal is to estimate some unknown function $f$, which serves as the information about the relationship between inputs and outputs. In Unsupervised Learning we observe only the predictors; the responses are not available. The goal of unsupervised learning is to discover something about the data, for example how it groups together. For this reason it is often referred to as Clustering. \\
	
	
	\noindent Supervised Learning is used for \textbf{inference} and \textbf{prediction}. In Prediction the goal is to estimate some unknown function f so as to accurately predict some output given a new input. This can be done for \textbf{quantitative} values, which is known as \textbf{regression} (on continuous values), or for \textbf{qualitative} (or categorical) values, which is referred to as \textbf{classification}.\\
	
	
	\noindent When talking about supervised learning it's also important to discuss \textbf{training data} and \textbf{test data}. The training data is used, as the name suggests, to train a statistical model. Test data are data that have not been given to the model during training and are used to test the performance of that model.\\
	
	
	\noindent The methods used to estimate $f$ can be categorized as \textbf{parametric} or \textbf{non-parametric}. Parametric methods make the assumption that the unknown function is linear, and use linear methods to estimate it. Non-parametric methods make no assumptions about the underlying form of $f$, and thus can use methods that are much more complicated and have many more degrees of freedom.\\
	
	\newpage
	\section*{Problem 2}
	
	
	Show that:
	
	\[ E(Y) = argmin_{c}E[(Y-c)^2] \]
	
	We are looking to proof that the value of $c$ for which $E[(Y-c)^2]$ attains its minimum is $E(Y)$.\\\\
	
	
	To show $E(Y) = argmin_cE[(Y-c)^2]$:
	\begin{itemize}
		\item $E[(Y-c)^2] = E[(Y-c)(Y-c)]$
		\item = $E[y^2 - 2Yc = c^2]$
		\item = $E[y^2] - 2cE[y] + c^2$	
		\item to minimize solve for where the gradient is zero: $\frac{d}{dc}E[(Y-c)^2] = 0 + 2E[y] + 2c$
		\item the value of $c$ where the gradient is zero is therefore: $c=E[y]$
		\item (optional)to verify this is a minimum take $\frac{d^2}{dc^2}$. The result is 2, which is greater than zero, therefore this value is a minimum.
	\end{itemize}
	
	
	
	
	\section*{Problem 3}
	
	Prove the bias-variance trade-off with irreducible error:\\\\
	
	
	
	$E[(y_{0} - \hat{f}(x_{0}) - E(\hat{f}(x_{0})))^2] + [E(\hat{f}(x_{0}) - f(x_{0})]^2 + Var(\epsilon) =$\\\\
	
	$Var(\hat{f}(x_{0})) + [Bias(\hat{f}(x_{0}))] + Var(\epsilon)$\\
	\\\\
	
	First, we expand $E[(y_{0} - \hat{f}(x_{0}))^2]$ into $E[y_{0}^2 + \hat{f}(x_{0})^2 - 2y_{0}\hat{f}(x_{0})]$\newline
    	\newline
  	Next, assuming that $y_{0}$ is deterministic, we can simplify to $E[\hat{f}(x_{0})^2] + y_{0}^2 - 2y_{0}E[\hat{f}(x_{0})]$\newline
    	\newline
    	With the rules of expectation we get $E[\hat{f}(x_{0})^2] - 2y_{0}E[\hat{f}(x_{0}) + y_{0}^2$\newline
    	\newline
    	Adding and Subtracting $E[\hat{f}(x_{0})]^2$ we get
	$E[\hat{f}(x_{0})^2] - E[\hat{f}(x_{0})]^2 + E[\hat{f}(x_{0})]^2  - 2y_{0}E[\hat{f}(x_{0})] + y_{0}^2$\newline
    	\newline
   	Next, $E[\hat{f}(x_{0})^2] - E[\hat{f}(x_{0})]^2 = Var(\hat{f}(x_{0}))$, so we get\newline 
    	\newline
    	$Var(\hat{f}(x_{0})) + E[\hat{f}(x_{0})]^2  - 2y_{0}E[\hat{f}(x_{0})] + y_{0}^2$\newline
    	\newline
    	Then we simplify to $Var(\hat{f}(x_{0})) + [E(\hat{f}(x_{0}) - y_{0})]^2$\newline
    	\newline
    	Then simplify again to $Var(\hat{f}(x_{0})) + [Bias(\hat{f}(x_{0}))]^2 + Var(\epsilon)$
	
	
	
	
	
	\section*{Problem 4}
	\subsection*{b}
	
	\paragraph{Data repartition} Train and testset.\\
	
	
	\begin{table}[H]
		\centering
		\begin{tabular}{|c|c|}
			\hline 
			& \textbf{Observations} \\ 
			\hline 
			Testset & 31 \\ 
			\hline 
			Trainset & 80 \\ 
			\hline 
			Total & 111 \\ 
			\hline 
		\end{tabular} 
	\end{table}


\paragraph{Data types}  There is four columns, in this work "ozone" is considered to be the output.\\

	\begin{table}[H]
		\centering
		\begin{tabular}{|c|c|}
			\hline 
			\textbf{Data} & \textbf{Data Type}\\ 
			\hline 
			ozone & Numerical \\ 
			\hline 
			radiation & Integer \\ 
			\hline 
			temperature & Integer \\ 
			\hline 
			wind & Numerical \\ 
			\hline 
		\end{tabular} 
	\end{table}
	
	
	\subsection*{c}
	
	 
	\begin{table}[H]
		\centering
		\begin{tabular}{|c|c|c|c|}
			\hline 
			& Range & Mean & SD \\ 
			\hline 
			Ozone & 1:168 & 42.099099 & 33.275969 \\ 
			\hline 
			Radiation & 7:334 & 184.801802 & 91.152302 \\ 
			\hline 
			Temperature & 57:97 & 77.792793 & 9.529969 \\ 
			\hline 
			Wind & 2.3:20.7 & 9.938739 & 3.559218 \\ 
			\hline 
		\end{tabular} 
	\end{table}
	
\subsection*{d}

The range of the Pearson correlation coefficient is between $-1$ and $1$. A coefficient of $1$ means that the variables have a total positive linear correlation, a $0$ means that there is no linear correlation and a $-1$ means that there is a total negative linear correlation. One can see in table \ref{tbl:pearson} that each variable has a total positive linear correlation with itself (obviously). \\

When we look at the graphs below and the Pearson correlations, we can see a relation between wind and ozone, that gives the impression of a negative sloped line and between temperature and ozone  but positively this time. The coefficients confirm what we see on the graph. The two related pairs present high coefficients. We have $-0.61$ for the wind-ozone pair and $0.69$ for the ozone-temperature pair. 
	
\begin{figure}[H]
	\centering
	\includegraphics[width=1\linewidth]{"img/ScatterplotByPair"}
	\caption{Scatter-plots between each variables.}
	\label{fig:scatterplotbypair}
\end{figure}
	


\begin{table}[H]
	\centering
	\label{tbl:pearson}
	\begin{tabular}{lllll}
		& ozone      & radiation  & temperature & wind       \\
		ozone       & 1.0000000  & 0.3483417  & 0.6985414   & -0.6129508 \\
		radiation   & 0.3483417  & 1.0000000  & 0.2940876   & -0.1273656 \\
		temperature & 0.6985414  & 0.2940876  & 1.0000000   & -0.4971459 \\
		wind        & -0.6129508 & -0.1273656 & -0.4971459  & 1.0000000 
	\end{tabular}
	\caption{Pearson Correlation between the features}
\end{table}
	
\subsection*{f}

Ozone prediction using a linear regression model. \\


\begin{figure}[H]
	\centering
	\includegraphics[width=0.7\linewidth]{"img/RplotOzone"}
	\caption{Predicted and true values of ozone for the test set}
	\label{fig:rplotozone}
\end{figure}


\textbf{RSS}: 8208.509\\

\textbf{Pearson} Correlation between predictions and true responses: 0.8268958. 




\subsection*{g}

 
\begin{figure}[H]
	\centering
	\includegraphics[width=0.7\linewidth]{"img/RplotRSS"}
	\caption{RSS for training and test set and for each k.}
	\label{fig:rplotrss}
\end{figure}


\begin{figure}[H]
	\centering
	\includegraphics[width=0.7\linewidth]{"img/RplotRSSNormalized"}
	\caption{Normalized RSS for training and test sets.}
	\label{fig:rplotrssnormalized}
\end{figure}




%  On which side of the graph do you have the most complex models?
The more complex models will be on the right hand side of the graph. This is due to the fact that the model will fit all the tiny changes in the data, each time a new point is added. Consequently it will lead to a high variance and a low bias. For this data, we would choose a k value around 6, minimising the RSS, as we can see on the graphs (normalized and not). \\

\noindent KNN doesn't make any assumption on the underlying data distribution. It is a non-parametric learning algorithm. 

% Argue with the bias-variance tradeof

% Which value of k would you choose for this data?

% In general, does the kNN method make any assumptions on the underlying data distribution?




	
	
\end{document}
