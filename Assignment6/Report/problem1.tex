\section*{Problem 1}

\paragraph{Question}How do you have to change the procedure of generating a MARS model to make a decision tree?

\paragraph{Answer} If you replace the piecwise linear basis functions used in MARS with step functions $I(x-t > 0)$ and $I(x-t < 0)$ and add a resriction that a node may not be split more than once, then you can make a descision tree. 

\paragraph{Question}Can you argue on the basis of the relationship between MARS and decision trees revealed in (a) what is an advantage of MARS over decision trees and what is an advantage of decision trees over MARS?
\\



\paragraph{Answer}For numeric data, MARS would tend to be better than Decision Trees because the reflected pairs (hinges) would adapt more accurately/simply to the underlying structure of the data (for instance local linearity) than the constant segmentation realised by decision trees. Think about a diagonal set of points that would have to be separated in many sub-trees, with MARS a single line and two knots can estimate it accurately. \\

\noindent The decision trees are more faster to create and, especially for categorical data, the fits of partitioning would be better. Even if MARS adapts to non linear and categorical data, depending on the exact shape of the data, a decision tree could separate more accurately the different regions.  





